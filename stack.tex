\documentclass[11pt,a4paper]{article}
\usepackage[T1]{fontenc}
\begin{document}
\title{Stack}
\author{Alföldi Mátyás}
\maketitle
\tableofcontents
\pagebreak
\section*{stack0}
\markright{}
\addcontentsline{toc}{section}{stack0}
This challange is relative simple, the following code is given:
\begin{verbatim}
int main(int argc, char **argv)
{
  volatile int modified;
  char buffer[64];
  modified = 0;
  gets(buffer);

  if(modified != 0) {
    printf("you have changed the 'modified' variable\n");
  } else {
    printf("Try again?\n");
  }
}
\end{verbatim}
I just gave the following input for this:\newline
10xa 10xb 10xc 10xd 10xe 10xf 10xg (So basically something that is longer than 64)
\section*{stack1}
\markright{}
\addcontentsline{toc}{section}{stack1}
Source code:\newline
https://web.archive.org/web/20170419031559/https://exploit-exercises.com/protostar/stack1/\newline
This time the argument has to be longer than 64, and 0x61626364 has to go into the variable, which is the ascii code for abcd, but because of the endianness has to be reversed.
\section*{stack2}
\markright{}
\addcontentsline{toc}{section}{stack2}
Source code:\newline
https://web.archive.org/web/20170419023252/https://exploit-exercises.com/protostar/stack2/\newline
Answer:\newline
Same as above but \textbackslash n \textbackslash r \textbackslash n \textbackslash r has to be used.
\section*{stack3}
\markright{}
\addcontentsline{toc}{section}{stack3}
Source code:\newline
https://web.archive.org/web/20170417130221/https://exploit-exercises.com/protostar/stack3/\newline
For this one we use objdump -d to get the address of the win function, which seems to be 08048424.\newline
Using /bin/echo -e -n '64 filler chars\textbackslash x24\textbackslash x84\textbackslash x04\textbackslash x08' | ./stack3 does the work
\section*{stack4}
\markright{}
\addcontentsline{toc}{section}{stack4}
Source code:\newline
https://web.archive.org/web/20170417130121/https://exploit-exercises.com/protostar/stack4/\newline
During the call to main the location to where the ret should return is pushed into the stack, so after checking with objdump the address of the win function a long enough input has to be generated so that we overwrite that.\newline
win is at: 080483f4\newline
Looking at main the following instructions modify esp:\newline
\begin{itemize}
\item push ebp
\item and esp, 0xfffffff0
\item sub esp, 0x50
\end{itemize}
And the following changes how long our filler string has to be:
\begin{itemize}
\item lea eax,[esp+0x10]
\item mov DWORD PTR [esp],eax
\end{itemize}
This way esp+0x10 gets passed to the gets function, and this means that the buffer is 64 long.\newline
The push ebp modifies esp by 4, and the and esp, 0xfffffff0 by a max of 15, so 64 + 15 + 4 should be the max len needed.\newline
So tests should be done with filler string len 68-84\newline
In the end 76 x a and 080484f4 reversed for the endiannes(in hex) was the amount needed to change where the ret brings us.\newline
(Used /bin/echo -e -n for writing the hex codes into the file)
\section*{stack5}
\markright{}
\addcontentsline{toc}{section}{stack5}
Source code:\newline
https://web.archive.org/web/20170419023355/https://exploit-exercises.com/protostar/stack5/\newline
This time we have to change where the ret brings us like before, but we also need to bring our shellcode to the stack, and somehow need to know where the address of it will be, to change where the ret brings us to that.\newline
The shellcode of our choice for the beginning 0xcc(int 3)\newline
Idea 1: after main-s ret there are 5 nop bytes, bringing execution to the first would be simple, only we would have to somehow overwrite that location.(And patching the location wouldn't be using shellcode.)
Idea 2: Check with gdb what the address for the ret is in the stack, overwrite that with its address+4, so that it will jump right after it, and place the shellcode there.\newline
(This will only work though, because aslr is disabled, and non executable locations, since the stack is usually non executable.)\newline
Decision: going with Idea 2\newline
We need 68-84 filler chars bffffd30 in hex, and also reversed and 0xcccccccc.\newline
76 filler chars are needed, afterwards 30fdffbfcccccccc (in hex) and if we use gdb stack5, then run <inputFile it stops at our cc.
\section*{stack6}
\markright{}
\addcontentsline{toc}{section}{stack6}
Source code:\newline
https://web.archive.org/web/20140405142902/http://exploit-exercises.com/protostar/stack6\newline
Here the return address, can't start with 0xbf(I guess it is the stack), so it has to be solved in another way.\newline
Idea 1 (Very simple example of return oriented programming): After the filler string modify the ret value to point to itself, then the next value will point to after itself, where the shell code will be located(int 3).\newline
Idea 2 (ret2libc): Because there is no aslr, I could also use print system, and search where I find the string /bin/sh or /bin/bash in memory, but this wouldn't work with aslr enabled.\newline
(print system outputs 0xb7ecffb0, and find for /bin/sh gives us 0xb7fba23f)\newline
So we have to place 0xb7ecffb0 first after the filler chars, then a value to which system should return, and lastly 0xb7fba23f\newline
\end{document}